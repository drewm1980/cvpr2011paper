\documentclass[times, 10pt,twocolumn, conference]{article}
\usepackage{cvpr}
\usepackage{times}
\usepackage{balance}
\usepackage{cite}
\usepackage{times}
\usepackage{epsfig}
\usepackage{subfigure}
\usepackage{graphicx}
\usepackage{times}
\usepackage{color}
\usepackage{url}
\usepackage{algorithm}
\usepackage{algorithmic}
\usepackage{latexsym, amsmath, amsfonts, amscd}
\usepackage{amssymb}

% Include other packages here, before hyperref.

% If you comment hyperref and then uncomment it, you should delete
% egpaper.aux before re-running latex.  (Or just hit 'q' on the first latex
% run, let it finish, and you should be clear).
%\usepackage[pagebackref=true,breaklinks=true,letterpaper=true,colorlinks,bookmarks=false]{hyperref}

\newtheorem{theorem}{Theorem}
\newtheorem{definition}[theorem]{Definition}
\newtheorem{principle}{Principle}
\newtheorem{problem}{Problem}
\newtheorem{property}{Property}
\newtheorem{example}{Example}
\newtheorem{remark}[theorem]{Remark}
\newtheorem{proposition}[theorem]{Proposition}
\def\arr{{\mathcal{A}}}
\newcommand{\Null}{{\mbox{Null}}}
\newcommand{\rank}{{\mbox{rank}}}
\renewcommand{\mathbf}{\boldsymbol}
\newcommand{\p}{{p}}
\renewcommand{\Re}{{\mathbb{R}}}
\newcommand{\Ce}{{\mathbb C}}
\newcommand{\Ze}{{\mathbb Z}}
\newcommand{\Pe}{{\mathbb P}}
\newcommand{\I}{\mathcal{I}}
\def\tt{{\boldsymbol{t}}}
\def\ee{{\boldsymbol{e}}}
\def\aa{{\boldsymbol{a}}}
\def\bb{{\boldsymbol{b}}}
\def\ss{{\boldsymbol{s}}}
\def\cc{{\boldsymbol{c}}}
\def\pp{{\boldsymbol{p}}}
\def\chat{{\boldsymbol{\hat{\c}}}}
\def\nn{{\boldsymbol{n}}}
\def\m{{\boldsymbol{m}}}
\def\X{{\boldsymbol{X}}}
\def\L{{\boldsymbol{L}}}
\def\l{{\boldsymbol{l}}}
\def\Y{{\boldsymbol{Y}}}
\def\V{{\boldsymbol{V}}}
\def\uu{{\boldsymbol{u}}}
\def\vv{{\boldsymbol{v}}}
\def\yy{{\boldsymbol{y}}}
\def\xx{{\boldsymbol{x}}}
\def\ww{{\boldsymbol{w}}}
\def\zz{{\boldsymbol{z}}}
\def\uu{{\boldsymbol{u}}}
\def\F{{\boldsymbol{F}}}
\def\veronese{{\nu}}
\def\Qveronese{{\mu}}
\def\H{{\mathcal{H}}}
\def\Q{{\mathcal{Q}}}
\def\E{{\mathcal{E}}}
\def\L{{\mathcal{L}}}
\def\M{{\mathcal{M}}}
\def\A{{\mathcal{A}}}
\def\B{{\mathcal{B}}}
\def\F{{\mathcal{F}}}
\def\field{\mathbb{F}}
\def\QED{~\rule[-1pt]{5pt}{5pt}\par\medskip}


\cvprfinalcopy % *** Uncomment this line for the final submission

\def\cvprPaperID{10 \vspace{-0.2in}} % *** Enter the CVPR Paper ID here
\def\httilde{\mbox{\tt\raisebox{-.5ex}{\symbol{126}}}}

% Pages are numbered in submission mode, and unnumbered in camera-ready
\ifcvprfinal\pagestyle{empty}\fi
\begin{document}

\twocolumn

%%%%%%%%% TITLE
\title{Accelerating $\ell_1$-Minimization Using Many-Core CPUs/GPUs \\ and Application to Face Recognition
\thanks{Corresponding author: . This work was partially supported by ARO MURI W911NF-06-1-0076.}}

\author{Anonymous\\
Paper ID: XXX
}

\maketitle
% \thispagestyle{empty}
%%%%%%%%% ABSTRACT
\begin{abstract}

\end{abstract}

%%%%%%%%% BODY TEXT
\section{Introduction}
\label{sec:introduction}
{\bf Drew, please draft this section.}

[Motivate the paper by an overview about the scope and importance of $\ell_1$-minimization].

[Review the available different algorithms solving $\ell_1$-minimization and $\ell_0$-minimization, please refer to Allen's SIAM paper.]

[Zero in and justify why we choose ALM for our implementation.]

[Further zero in and limit the application to face recognition. What are the special data structure. What are the selected hardware architecture. Why it is important to consider both CPU and GPU implementation (difference in cores and caches)].

\subsection{Literature review}
[Review the past parallel algorithms for $\ell_1$-minimization. Hopefully there aren't many]

\subsection{Contributions of the paper}
[Can leave empty until later.]

\section{Parallelization using Augmented Lagrange Multiplier Methods}
{\bf Victor and Mark, please draft this section.}

[An discussion of the original ALM algorithm should go first]

[Parallelization steps go here]

[A discussion about computational complexity and memory cost should go here]

\section{Implementation in Many-Core CPU/GPU Architectures}
\label{sec:algorithm}

{\bf Victor and Drew, please draft this section}

\subsection{CPU Implementation}

\subsection{GPU Implementation}

\section{Experiment}
\label{sec:experiment}

{\bf Drew, please draft this section}

\section{Conclusion}
\label{sec:conclusion}

{\small
\bibliographystyle{ieee}
\bibliography{paper}
}

\end{document}
