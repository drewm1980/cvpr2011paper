\documentclass[10pt,twocolumn,letterpaper]{article}

\usepackage{../authorkit/cvpr}
\usepackage{times}
\usepackage{epsfig}
\usepackage{graphicx}
\usepackage{amsmath}
\usepackage{amssymb}

% Packages added by us
\usepackage{mydefs}
\usepackage{algorithm}
\usepackage{algorithmic}

\usepackage[pagebackref=true,breaklinks=true,letterpaper=true,colorlinks,bookmarks=false]{hyperref}

% \cvprfinalcopy % *** Uncomment this line for the final submission

\def\cvprPaperID{****} % *** Enter the CVPR Paper ID here
\def\httilde{\mbox{\tt\raisebox{-.5ex}{\symbol{126}}}}

% Pages are numbered in submission mode, and unnumbered in camera-ready
\ifcvprfinal\pagestyle{empty}\fi
\begin{document}

\title{Efficient Parallelization of Sparse Representation for Face Recognition}

\author{First Author\\
Institution1\\
Institution1 address\\
{\tt\small firstauthor@i1.org}
% For a paper whose authors are all at the same institution,
% omit the following lines up until the closing ``}''.
% Additional authors and addresses can be added with ``\and'',
% just like the second author.
% To save space, use either the email address or home page, not both
\and
Second Author\\
Institution2\\
First line of institution2 address\\
{\small\url{http://www.author.org/~second}}
}

\maketitle

\begin{abstract}
   The ABSTRACT is to be in fully-justified italicized text, at the top
   of the left-hand column, below the author and affiliation
   information. Use the word ``Abstract'' as the title, in 12-point
   Times, boldface type, centered relative to the column, initially
   capitalized. The abstract is to be in 10-point, single-spaced type.
   Leave two blank lines after the Abstract, then begin the main text.
   Look at previous CVPR abstracts to get a feel for style and length.
\end{abstract}

\section{Introduction}

\section{Application to Face Recognition}
\begin{algorithm}[thb]
\caption{\bf (Deformable Sparse Recovery and Classification for
Face Recognition)} \label{alg:deformable-src}
\begin{algorithmic}[1]
\STATE {\bf Input:} Frontal training images $A_i \in \Re^{m\times n_i}, i=1,2,\ldots,K$ for $K$ subjects,  a test image
$\y\in\Re^m$ and a deformation group $T$. 
\STATE
{\bf for} each subject $i$, 
\STATE \hspace{3mm} $\tau^{(0)}
\leftarrow I$. 
\STATE \hspace{3mm} {\bf while} not converged $(j=1,2,\ldots)$ {\bf do} 
\STATE \hspace{6mm}
$\tilde \y(\tau) \leftarrow \frac{\y \circ \tau}{\|\y \circ
\tau\|_2}$; \;\;\; $J \leftarrow  \frac{\partial}{\partial
\tau} \tilde\y(\tau)  \bigr|_{\tau^{(j)}} $;
%\STATE \hspace{6mm} $(\hat \x, \hat \e, \Delta \tau) \leftarrow \left\{\begin{array}{l} \arg \min_{\x,\e,\Delta \tau} \| \e \|_1 \\  \subj \; \y + J \Delta \tau = A_k \x + \e \end{array}\right.$
\STATE \hspace{6mm} $ \Delta \tau =  \arg\min \; \| \e \|_1  \;
\subj \; \tilde \y + J \Delta \tau = A_i \x + \e.$ 
\STATE
\hspace{6mm} $\tau^{(j+1)} \leftarrow \tau^{(j)} + \Delta
\tau$; 
\STATE \hspace{3mm} {\bf end while} \STATE {\bf end} \STATE Keep
the top $S$ candidates $k_1, \ldots, k_S$ with the smallest
residuals $\|\e\|_1$. \STATE Compute an average transformation
$\bar{\tau}$ from $\tau_{k_1}, \tau_{k_2}, \ldots, \tau_{k_S}$.
\STATE Update $\y \leftarrow \y \circ \bar{\tau}$ and $\tau_i
\leftarrow \tau_i \cdot \bar{\tau}^{-1}$ for $i = k_1, \dots,
k_S$. \STATE Set $A \leftarrow \big[ A_{k_1} \circ
\tau_{k_1}^{-1} \mid A_{k_2} \circ \tau_{k_2}^{-1} \mid \dots
\mid A_{k_S} \circ \tau_{k_S}^{-1} \big]$. \STATE Solve the
$\ell^1$-minimization problem:
$$\hat{\x} = \arg \min_{\x, \e} \| \x \|_1 + \|\e\|_1 \;\; \subj \;\; \y = A \x + \e.$$
\STATE Compute residuals $r_i(\y) = \| {\y} - {A}_i \, \delta_i(\hat{\x}) \|_2$ for $i = k_1, \dots, k_S$.
\STATE {\bf Output:} $\mbox{identity}(\y) = \arg\min_i r_i(\y)$.
\end{algorithmic}
\end{algorithm}


{\small
\bibliographystyle{ieee}
\bibliography{faces}
}


\end{document}
